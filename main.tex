% Copyright (c) 2023 Boshi Yuan
% Copyright (c) 2024 yueneiqi
% 
% This software is released under the MIT License.
% https://opensource.org/licenses/MIT

\documentclass[a4paper,zihao=-4,AutoFakeBold]{ctexart}

% 实际被 \newgeometry 分为两部分控制,搜索该命令进行修改
\usepackage[top=2.5cm,bottom=2.5cm,left=2.8cm,right=2.8cm]{geometry}
\usepackage{fontspec}
\usepackage[shortlabels]{enumitem}
\usepackage{amsmath}
\usepackage{amssymb}
\usepackage{titlesec}
\usepackage{graphicx}
\usepackage{array}
\usepackage{makecell}
\usepackage[tikz]{mdframed}
% 可使用 style=gb7714-2015 选项来指定参考文献国标样式,但官方模板对此未做要求
% 使用 sorting=none 选项指定按照文章内引用顺序来排序参考文献
\usepackage[maxbibnames=9,sorting=none]{biblatex} 
\addbibresource{ref.bib}
\usepackage{etoolbox}
\usepackage[hidelinks]{hyperref}
\usepackage{zhlipsum}

% 使用类似 Word 上的伪粗宋体
\newCJKfontfamily\bsongti{SimSun.ttc}[AutoFakeSlant,AutoFakeBold]
% 使用类 Times 西文字体
\setmainfont{texgyretermes}[
    Extension      = .otf ,
    UprightFont    = *-regular,
    BoldFont       = *-bold,
    ItalicFont     = *-italic,
    BoldItalicFont = *-bolditalic,
]
% 你也可以把上面的字体设置命令去掉,改为下面的 Times New Roman
% \setmainfont{Times New Roman}

% % 绘制 ``研究课题来源'' 表格里用到的勾和框
% \myunchecked 绘制未打勾的方框
\newcommand{\myunchecked}{%
    \tikz[baseline] \draw[semithick] (1pt,1pt) rectangle (6pt,6pt);%
}
\NewDocumentCommand{\mychecked}{s}{%
    \IfBooleanTF{#1}{% \mychecked* 绘制一个稍大的勾
        \tikz[baseline]{
            \draw (-5.5pt,5pt) -- (-3.46pt,6pt); 
            \draw[thick] (-3.46pt,6pt) -- (0,0);
            \draw (0,0) -- (6.92pt,12pt);
        }%
    }{% \mychecked 绘制一个稍小的勾
        \tikz[baseline]{
            \draw[semithick] (1pt,2pt) -- (1.66pt,2.66pt); 
            \draw[thick] (1.66pt,2.66pt) -- (3.66pt,0);
            \draw[semithick] (3.66pt,0) -- (6pt,6pt);
        }%
    }%
}

% 控制表格的线框粗细
\def\mymfdlinewidth{0.5pt}
\def\mymfdsplitlinewidth{0.8pt}

% 一级标题仿宋、左对齐
\titleformat{\section}[block]{\zihao{-4}\fangsong}{\thesection、}{0pt}{\linespread{1.0}\selectfont}
% 一级三级小标题楷体,与正文字体一致
\titleformat*{\subsection}{\kaishu}
\titleformat*{\subsubsection}{\kaishu}

\newcommand{\splitframe}{
    \mdfsubtitle[
        subtitleaboveline=true,
        subtitleabovelinewidth=\mymfdsplitlinewidth,
        % subtitleaboveskip=0.5\baselineskip,
        subtitlebelowskip=0.2\baselineskip,
        subtitleinneraboveskip=0pt,
        subtitleinnerbelowskip=0pt,
    ]{}
}

\begin{document}

\pagestyle{empty}

% 首页的页边距单独设置
\newgeometry{top=2.5cm,bottom=2.5cm,left=2.45cm,right=2.45cm}

\vspace{-10ex} % 不起作用,Latex会省略掉全文第一个vspace
\vspace*{11.3ex}  % 启用

\begin{center}
    \vspace{-0.5cm}
    {\zihao{1}\bfseries\bsongti 华东师范大学}\\~\\
    {\zihao{-3}\bfseries East China Normal University}\\~\\
    {\zihao{1}\bfseries\bsongti 学位论文开题报告登记表}\\~\\
    {\zihao{-3}\bfseries Thesis/Dissertation Proposal Form}
    \vspace{0.4cm}
\end{center}


%%%%%%%%%%%%%%%%%%%%%%%%%%%%%%%%%%%%%%%%%%%%%%%%%%%%%%%%%%%
% 填写说明:请在下面表格中填写你的信息,注意事项:
% 1. 论文题目如果太长,使用 \\ 手动换行 (见示例)
% 2. 本模板提供了 \mychecked 命令用来打勾,
%    \myunchecked 命令用来画出未打勾的方框,
%    你也可以自己使用 LaTeX 各种宏包里提供的打勾和方框命令代替
%%%%%%%%%%%%%%%%%%%%%%%%%%%%%%%%%%%%%%%%%%%%%%%%%%%%%%%%%%%
\begin{table}[h]
    \renewcommand{\arraystretch}{1.85}
    % 左栏中文为楷体加粗四号字,英文为小四加粗;右栏为仿宋五号加下划线
    \zihao{5}
    \begin{tabular}{>{\bfseries\kaishu}p{5.7cm}l>{\songti}l}
        {\zihao{4}学号}~~{\zihao{-4}Student No.}
            & 00111110111 \\\cline{2-2}
        {\zihao{4}姓名}~~{\zihao{-4}Name}
            & 师范大学 \\\cline{2-2}
        {\zihao{4}学生类别}~~{\zihao{-4}Degree Program}
            & \makecell*[l]{
            \myunchecked\ 学术学位博士生 Academic Doctoral Student\\
            \myunchecked\ 专业学位博士生 Professional Doctoral Student\\
            \myunchecked\ 学术学位硕士生/Academic Master Student\\
            \mychecked\ 专业学位硕士生/Professional Master Student
        }\vspace{-1.5ex}\\ % 与官方 word 版仔细比对后选取的经验值
        {\zihao{4}学习形式}~~{\zihao{-4}Student Mode}
            & \makecell[l]{
            \myunchecked\ 全日制/Full-time\\
            \mychecked\ 非全日制/Part-time
        }\vspace{-1.5ex}\\ % 与官方 word 版仔细比对后选取的经验值
        {\zihao{4}导师}~~{\zihao{-4}Supervisor(s)}
            & 我的导师 \\\cline{2-2}
        {\zihao{4}论文题目}~~{\zihao{-4}Thesis Title}
            & \makecell[l]{我很长很长很长很长很长很长很长很长很长的 \\ 
                            很厉害的论文题目}\\\cline{2-2}
        {\zihao{4}专业/领域}~~{\zihao{4}Major}
            & 我的专业 \\\cline{2-2}
        {\zihao{4}培养单位}~~{\zihao{4}School}
            & 我的学院 \\\cline{2-2}
    \end{tabular}
    \vspace{-5cm}   % 防止表格过长被排到下一页
\end{table}

\clearpage
% 正文的页边距
\newgeometry{top=2.5cm,bottom=2.5cm,left=2.8cm,right=2.8cm}


\begin{center}
    \vspace*{-0.5cm}
    {\zihao{-3}\heiti 填\quad 写\quad 说\quad 明}\\~\\
    \vspace*{-0.5ex}
    {\zihao{-3}\bfseries Instruction}
\end{center}

\vspace*{0.3ex}
\setlength{\leftmargini}{21pt}
\begin{enumerate}[1、]
    \fangsong
    \item 开题报告为A4大小,于左侧装订成册。各栏空格不够时,请自行加页。
          \\[0.5\baselineskip]
          This form should be printed on A4 paper and bound up on the left. In case the space 
          left is not enough, please feel free to add extra pages.

    \vspace{3\baselineskip}
    \vspace{-0.3ex}

    \item 开题报告通过后,分别由研究生、导师、学院各存档一份。
          \\[0.5\baselineskip]
          After the dissertation proposal is accepted, three copies of this form should be filed, 
          one for the student, one for the supervisor, and one for school.

\end{enumerate}

\clearpage

% 正文格式
\zihao{-4}\fangsong\linespread{1.09}\selectfont % 与官方 word 版仔细比对后选取的经验值

\mdfdefinestyle{mymdfstyle}{
    everyline=true,
    linewidth=\mymfdlinewidth,
    innerleftmargin=4pt,
    innerrightmargin=4pt,
    innertopmargin=-15pt,
    innerbottommargin=-10pt
}
\begin{mdframed}[
    style=mymdfstyle,
]

%%%%%%%%%%%%%%%%%%%%%%%%%%%%%%%%%%%%%%%%%%%%%%%%%%%%%%%%%%%%%
% 填写说明:开始撰写你的开题报告。
% 1. section 里是开题报告模板里的填写要求,已为你准备好。
% 2. 请接在后面填写你的正文内容,具体请仔细阅读示例和注释。
% 3. 你可以使用 \subsection 和 \subsubsection 命令来使用小标题
%    小标题都可以使用 \label 打标签,并用 \ref 进行引用。
% 4. 表格内的想另起一页使用 \newpage,增加单元格使用 \splitframe
%%%%%%%%%%%%%%%%%%%%%%%%%%%%%%%%%%%%%%%%%%%%%%%%%%%%%%%%%%%%%

\vspace{-0.5ex} % 与官方 word 版仔细比对后选取的经验值
\section{选题的研究意义及国内外研究现状综述,附参考文献。\newline
Significance of the research topic, overview of the current research status, and the works cited.}

让我来综述一下课题国内外研究进展、现状、挑战与意义。
这里要写好多好多字,博士生要写一万字,硕士生要写五千字,
所以我们要多引用一些文献来达到字数要求。
让我们引用著名的SPDZ协议\cite{SPDZ}。
当然我们也可以引用一个中文文献\cite{ZJSD}。
让我们写一点数学公式 $c^2 = a^2 + b^2$。

你还可以使用二级标题和三级标题,
并使用\LaTeX 的 \verb|\label| 和 \verb|\ref| 命令进行引用,
如第\ref{sub}节和第\ref{subsub}节所示。


\subsection{二级标题}\label{sub}

% 文字来自《祝福》,以展示排版效果
旧历的年底毕竟最像年底,村镇上不必说,就在天空中也显出将到新年的气象来。
灰白色的沉重的晚云中间时时发出闪光,接着一声钝响,是送灶的爆竹;
近处燃放的可就更强烈了,震耳的大音还没有息,空气里已经散满了幽微的火药香。
我是正在这一夜回到我的故乡鲁镇的。


\subsubsection{三级标题}\label{subsub}

虽说故乡,然而已没有家,所以只得暂寓在鲁四老爷的宅子里。
他是我的本家,比我长一辈,应该称之曰“四叔”,是一个讲理学的老监生。
他比先前并没有甚么大改变,单是老了些,但也还未留胡子,
一见面是寒暄,寒暄之后说我“胖了”,说我“胖了”之后即大骂其新党。
但我知道,这并非借题在骂我:因为他所骂的还是康有为。
但是,谈话是总不投机的了,于是不多久,我便一个人剩在书房里。

最后要打印出参考文献列表,在列出列表之前,我空了2个空行出来。

\vspace{2\baselineskip}

{
\renewcommand{\bibfont}{\small}
\noindent 参考文献 References: 
    \printbibliography[heading=none]

}

%下面两种方式二选一
\newpage %另起一页
% \splitframe %不另起页只分框

\section{课题研究目标、主要研究内容和拟解决的关键问题。\newline
    Research objectives, main content and key issues to be solved.}

\zhlipsum %删除此行,从这里开始写

%下面两种方式二选一
\newpage %另起一页
% \splitframe %不另起页只分框

\vspace{-0.5ex} % 与官方 word 版仔细比对后选取的经验值
\section{拟采取的研究方法、技术路线、实验方案及其可行性分析。(研究方法涉及到\break
人体或动物实验的,需首先进行伦理审查,审查通过后,须将人体、动物实验伦\vspace{0.3ex}\break % 与官方 word 版仔细比对后选取的经验值
理审查意见作为开题报告的一部分,并同时附伦理审查通过批准函)\newline
Research methods, technical route, experimental scheme to be adopted and its feasibility analysis.}

\zhlipsum %删除此行,从这里开始写

%下面两种方式二选一
\newpage %另起一页
% \splitframe %不另起页只分框

\vspace{-0.5ex} % 与官方 word 版仔细比对后选取的经验值
\section{选题的创新性或应用性。Novelty or applicability of the proposed topic.}

\zhlipsum %删除此行,从这里开始写

%下面两种方式二选一
% \newpage %另起一页
\splitframe %不另起页只分框

\vspace{-0.5ex} % 与官方 word 版仔细比对后选取的经验值
\section{论文大纲。Thesis outline.}

\setlist[enumerate]{itemsep=0pt,parsep=0pt,partopsep=0pt,topsep=0pt}
\setlist[enumerate,1]{label=第\chinese*章,itemindent=20pt,leftmargin=50pt}
\setlist[enumerate,2]{label=\arabic{enumi}.\arabic*,itemindent=-10pt}
\setlist[enumerate,3]{label=\arabic{enumi}.\arabic{enumii}.\arabic*,itemindent=-10pt}

\begin{enumerate}
    \item 引言
    \item 相关理论与技术
    \begin{enumerate}
        \item 二级标题
            \begin{enumerate}
                \item 三级标题
            \end{enumerate}
    \end{enumerate}
    \item 系统设计
    \item 实验
    \begin{enumerate}
        \item 实验环境
        \item 测试结果
    \end{enumerate}
    \item 总结与展望
    \begin{enumerate}
        \item 工作总结
        \item 工作展望
    \end{enumerate}
\end{enumerate}

\setlist[enumerate]{}
\setlist[enumerate,1]{}
\setlist[enumerate,2]{}
\setlist[enumerate,3]{}

%下面两种方式二选一
% \newpage %另起一页
\splitframe %不另起页只分框

\vspace{-0.5ex} % 与官方 word 版仔细比对后选取的经验值
\section{研究计划进度、预期成果。Research schedule and expected outcomes.}

\zhlipsum %删除此行,从这里开始写

%下面两种方式二选一
% \newpage %另起一页
\splitframe %不另起页只分框

\vspace{-0.5ex} % 与官方 word 版仔细比对后选取的经验值
\section{与本课题有关的工作积累、已有的研究工作成绩。Previous work in relation to the proposed topic.}

目前已经积累了很多很多工作了,也取得了非常非常好的工作成绩。 %删除此行,从这里开始写

%下面两种方式二选一
% \newpage %另起一页
\splitframe %不另起页只分框

\vspace{0.7ex} % 与官方 word 版仔细比对后选取的经验值
\noindent
本人承诺:开题报告中的内容真实无误,若有不实,愿承担相应的责任和后果。
I hereby declare and affirm that the content provided in this Form is valid and accurate\break
and that I shall bear the corresponding liabilities and consequences for any untrue\break
statement found.

\vspace{\baselineskip}

\noindent
学生签字/Signature of Student: 
\hfill              
日期/Date:202Y-MM-DD
\vspace{-0.5ex}  % 与官方 word 版仔细比对后选取的经验值
\\

% 新版请取消下面注释
% \vspace{-5ex} % 与官方 word 版仔细比对后选取的经验值
% \splitframe
% \vspace{0.7ex} % 与官方 word 版仔细比对后选取的经验值

% \noindent
% 导师意见Comments of Supervisor:

% \vspace{2\baselineskip}

% \noindent
% 导师签字/Signature of Supervisor: 
% \hfill              
% 日期/Date:202Y-MM-DD
% \vspace{2ex} % 与官方 word 版仔细比对后选取的经验值
% \\

\end{mdframed}

\end{document}
